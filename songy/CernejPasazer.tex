\begin{song}{title=\predtitle\centering Černej Pasažér \\\large Traband \vspace*{-0.5cm}}  %% sem se napíše jméno songu a autor

\moveright -.5cm \vbox{
\begin{centerjustified}
\nejvetsi

\begin{varwidth}[t]{0.48\textwidth}\setlength{\parindent}{\pindent}  %Varianta č. 2 --> Dva sloupce
\nejvetsi

\setcounter{Slokočet}{0}
\sloka
Mám ^{Dmi\z}kufr~plnej přebytečnejch

^{A\z }krámů

a mapu zabalenou do ^{Dmi}plátna.

Můj vlak však jede na opačnou

^{A\z }stranu

a moje jízdenka je dávno ^{Dmi\z }neplatná.

\mezera
\textbf{F  Dmi F  Dmi }

\sloka
Někde ve vzpomínkách stojí dům.

Ještě vidím, jak se kouří z komína.

V tom domě prostřený stůl,

tam já a moje rodina.

\sloka
Moje minulost se na mě šklebí

a srdce bolí, když si vzpomenu,

že stromy, který měly dorůst

k nebi,

teď leží vyvrácený z kořenů.

\mezera
\textbf{F  Dmi F  Dmi }

\refren
Jsem černej ^{B\z }pasažér,

^{C\z }nemám ^{F}cíl ani směr,

vezu se ^{B\z }načerno ^{C\z }životem a ^{F\z }nevím.

Jsem černej ^{B\z }pasažér,

^{C\z }nemám ^{F}cíl ani směr,

vezu se ^{B\z }odnikud ^{C\z }nikam a ^{A7\z }nevím,

kde skončím.

\end{varwidth}\mezisloupci \begin{varwidth}[t]{0.48\textwidth}\setlength{\parindent}{\pindent}  % V případě varianty č.2 jde odsud text do pravé části
\vspace*{0.41cm}
\sloka
Mám to všechno na barevný fotce

někdy z minulýho století.

Tu jedinou a pocit bezdomovce

si nesu s sebou jako prokletí.

\mezera
\textbf{F  Dmi F  Dmi }

\refren

\sloka
Mám kufr plnej přebytečnejch

krámů

a mapu zabalenou do plátna.

Můj vlak však jede na opačnou

stranu

a moje jízdenka je dávno neplatná.

\refren

\sloka
Ze Smíchovskýho nádraží

rychlík mě někam odváží.

Srdce jak těžký závaží

ze Smíchovskýho nádraží\elipsa\dots

\end{varwidth}   %Součást druhé varianty

\end{centerjustified}
}
\end{song}
\setcounter{Slokočet}{0}
