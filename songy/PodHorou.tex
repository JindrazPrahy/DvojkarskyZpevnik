\begin{song}{title=\predtitle\centering Pod horou \\\large  František Vrba \& Jim Čert  \vspace*{-0.3cm}}  %% sem se napíše jméno songu a autor

\moveleft 0.5cm \vbox{

\begin{centerjustified}
\begin{varwidth}[t]{0.58\textwidth}\setlength{\parindent}{\pindent}  %Varianta č. 2 --> Dva sloupce
\vspace*{0.19cm}
\sloka
^{Hmi \z}Dřív~nežli vzejde ^{A \z}světlo dne,

přes horstvo, jež se ^{Hmi \z}v~mlze pne,

jdem do hlubin, kde ^{A \z}vládne stín,

hledat své zlato ^{Hmi \z}kouzelné.

\sloka
Znal naše kouzla zemský klín,

když rod náš v třesku kovadlin

kul klenoty a temnoty

zaháněl v slujích, kde spal stín.

\sloka
A mnohý elf i dávný král

měl od nás meč, co zářně plál,

když tě  náš um těm vladařům

do jílců oheň včaroval.

\sloka
Dali jsme stříbru hvězdný třpyt,

korunám zlatým slunce svit --

tu krásu krás a skvělý jas

jsme předli z drátků jako nit.

\sloka
A co jsme měli pohárů

a zlaté harfy postaru,

jenže náš zpěv člověk i elf

neslýchal z hlubin, ani hru.

\sloka
Zvon v údolí bil na poplach

a lidem zbělel tváře strach,

když dračí spár hůř nežli žár

jim pohřbil město v sutinách.

\end{varwidth}\mezisloupci\begin{varwidth}[t]{0.41\textwidth}\setlength{\parindent}{\pindent}
\vspace*{0.60cm} % V případě varianty č.2 jde odsud text do pravé části

\sloka
Dýmala hora pod lunou,

v tu chvíli pro nás osudnou,

každý se hnal, než drak do skal

pod svými drápy, pod lunou.

\sloka
Dřív nežli vzejde slunce svit,

přes chmurný, mlžný horský štít

jdem do hlubin, kde vládne stín,

mu harfy své i zlato vzít.

\end{varwidth}

\end{centerjustified}
}

\setcounter{Slokočet}{0}
\end{song}
