\begin{song}{title=\centering Už to nenapravím \\\normalsize Jaroslav Samson Lenk  \vspace*{-0.3cm}}  %% sem se napíše jméno songu a autor
\moveright 2cm \vbox{      %Varianta č. 1  ---> Jeden sloupec zarovnaný na střed	

\refren /: ^{Cm}vap tada dap \dots ^{F\,As\,G\,G7} :/

\sloka
	V ^{Ami}devět hodin dvacet pět mě ^{F}opustilo štěstí, 
	
	ten ^{A}svlak, co jsem jím měl jet, na koleji ^{G}dávno ^{G7}nestál.
	
	V ^{C}m devět hodin dvacet pět ^{F}jako bych dostal pěstí,
	
	já ^{As}za hodinu na náměstí měl jsem ^{G}stát, ale ^{G7}v jiným městě.
	
	
	Tvá ^{C}zpráva zněla prostě a ^{C7}byla tak krátká,
		
	že ^{Fmi}stavíš se jen na skok, že nechalas mi vrátka ^{H}zadní otevřená, ^{G}zadní otevře^{G7}ná.
	
	Já ^{C}naposled tě viděl ^{C7}když ti bylo dvacet 
	
	a ^{Fmi}to si tenkrát řekla, že už se nechceš vracet, ^{H}že si unavená, ^{G}ze mě unave^{G7}ná.
	


\refren

\sloka
	Já ^{Cmi}čekala jsem, hlavu jako střep a ^{F}zdálo se, že dlouho, 
	
	snad ^{As}může za to vinný sklep, že člověk ^{G}často ^{G7}sleví.
	
	Já ^{Cmi}čekala jsem, hlavu jako střep, ^{Fs}podvědomou touhou, 
	
	já ^{As}čekala jsem dobu dlouhou víc než ^{G}dost, kolik ^{G7}přesně nevím.


	Pak ^{C}jedenáctá bila a ^{C7}už to bylo passé, 

	já ^{Fmi}dřív jsem měla vědět, že vidět tě chci zase, že ^{H}láska nerezaví, ^{G}láska nereza^{G7}ví.
	
	Ten ^{C}dopis, co jsem psala byl ^{C7}dozajista hloupý,
	
	byl ^{F}modměřený moc, na vlídný slovo skoupý, ^{H}už to nenapravím, ^{G}už to nenapra^{G7}vím.
	


\refren



}
\setcounter{Slokočet}{0}
\end{song}
