%%%%%%%%%%%%%%%%%%%%%%%%%%%%%%%%%%%%%%%%%%%%%%%%%%%%%
%			ŠABLONA PÍSNIČEK v. 18.09               %
%%%%%%%%%%%%%%%%%%%%%%%%%%%%%%%%%%%%%%%%%%%%%%%%%%%%%
% Tento soubor slouží jako (naučná) šablona, pomocí 
% které lze vytvářet zdrojové soubory k jednotlivým 
% písním.
%%%%%%%%%%%%%%%%%%%%%%%%%%%%%%%%%%%%%%%%%%%%%%%%%%%%%
%			Jak psát soubory songů?                 %
%%%%%%%%%%%%%%%%%%%%%%%%%%%%%%%%%%%%%%%%%%%%%%%%%%%%%
%	1. Text písně se začíná psát na místě START 
%	   a končí na místě END. Zbylý text ignorujte.
%	2. Jak bude vypadat pdf písně zjistíte po tom, 
%	   co soubor zkompilujete pomocí souboru   
%      ../Generator/generator. 
%	3. Při psaní dodržujte následující TeX pravidla:
%	 a) Nový řádek napíšete pomocí dvou odsazení 
%	    tedy dvou enterů.
%	 b) Nová sloka se píší pomocí \sloka a odsazení.
%		Refrén se píše jako \refren, v případě více 
%		refrénů \refren[č. refrénu].
%	 c) Akordy se píšou tak, že napíšete před slovo,
%	    kde chcete mít akord (bez mezery):
%		^{AKORD1\,AKORD2...}.
%	4. Pokud chcete ušetřit tvůrcům práci, tak 
%	   si přečtěte další poučný soubor o typografii 
%	   ../../Typo_pravidla.txt.
%	5. Akordy stačí psát jen do první sloky, když 
%	   se nezmění -- kytaristé to zvládnou
%	7. Název písně pište na místo [NÁZEV] a autora 
%	   pište na místo [AUTOR] 
%	7. Jak psát věci na české klávesnici:
%	   \ = alt gr + q; [/] = alt gr f/g; 
%      {/} = alt gr + b/n; ^ = alt gr + 3 , cokoliv
%%%%%%%%%%%%%%%%%%%%%%%%%%%%%%%%%%%%%%%%%%%%%%%%%%%%%
%			Jak kompilovat jednotlivé písně?        %
%%%%%%%%%%%%%%%%%%%%%%%%%%%%%%%%%%%%%%%%%%%%%%%%%%%%%
%	1. Více návodu je k tomuto napsáno v souboru 
%      ../Generator/generator. 
%%%%%%%%%%%%%%%%%%%%%%%%%%%%%%%%%%%%%%%%%%%%%%%%%%%%%
%			Jak kompilovat celý zpěvník?			%
%%%%%%%%%%%%%%%%%%%%%%%%%%%%%%%%%%%%%%%%%%%%%%%%%%%%%
%	1. Více návodu je k tomuto napsáno v souboru
%	   ../Cely_zpevnik/zpevnik.tex.
%%%%%%%%%%%%%%%%%%%%%%%%%%%%%%%%%%%%%%%%%%%%%%%%%%%%%
\begin{song}{title=\predtitle \centering Nad stádem koní \\\large Buty }  %% sem se napíše jméno songu a autor

\vspace*{.5cm}

\begin{centerjustified}
\begin{varwidth}[t]{0.48\textwidth}\setlength{\parindent}{\pindent}  %Varianta č. 2 --> Dva sloupce
\vetsi
\sloka
^{D\z}Nad stádem ^{A\z}koní ^{Emi}

podkovy ^{G\z}zvoní, ^{\z D}zvoní,

černý vůz ^{A\z}vlečou ^{Emi}

a slzy ^{G\z}tečou a já volám.

\sloka
Tak neplač, můj kamaráde,

náhoda je blbec, když krade,

je tuhý jak veka a řeka ho zplaví,

máme ho rádi.

\refren
No tak ^{C\z}co, tak ^{G\z}co, tak ^{A\z}co.

\sloka
Vždycky si přál, až bude popel

i s kytarou, hou,

vodou ať plavou, jen žádný hotel

s křížkem nad hlavou.

\sloka
Až najdeš místo, kde je ten pramen

a kámen, co praská,

budeš mít jisto, patří sem popel

a každá láska.

\refren

\sloka
Nad stádem koní

podkovy zvoní, zvoní,

černý vůz vlečou

a slzy tečou a já šeptám.

\sloka
Vysyp ten popel, kamaráde,

do bílé vody, vody,

vyhasnul kotel a náhoda je

štěstí od podkovy.

\end{varwidth}\mezisloupci\begin{varwidth}[t]{0.5\textwidth}\setlength{\parindent}{\pindent}
\vspace*{0.60cm}  % V případě varianty č.2 jde odsud text do pravé části

\refren[2]
\textbf{G D G D G}

Vysyp ten ^{D\z}popel ^{A\z G}kamaráde

(Heja hej\elipsa.\elipsa.\elipsa.)

do bílé ^{D\z}vody, ^{A\z G}vody.

(Heja hej\elipsa.\elipsa.\elipsa.)

Vyhasnul ^{D\z}kotel a ^{A\z }náhoda ^{Emi \z}je~~~~~~

(Heja hej\elipsa.\elipsa.\elipsa.)

štěstí od ^{\z G}podkovy.

(Heja hej\elipsa.\elipsa.\elipsa.)

\refren[2] $2\times$

\mezera

\includefret{D}
\includefret{A}

\includefret{Emi}
\includefret{G}

\includefret{C}

\end{varwidth}
\end{centerjustified}
\setcounter{Slokočet}{0}
\end{song}
