%\documentclass[../main.tex]{subfiles}

\begin{song}{title=\centering Traktor \\\normalsize Visací zámek \vspace*{-0.3cm}}  %% sem se napíše jméno songu a autor

% \begin{minipage}[t]{0.2\textwidth}\setlength{\parindent}{0.45cm}  %Varianta č. 2 --> Dva sloupce
% \vspace*{-12cm}
% \centering
% \includegraphics[width = 2.2cm]{../Akordy/f.png}

% \includegraphics[width = 2.2cm]{../Akordy/gxver2.png}

% \includegraphics[width = 2.2cm]{../Akordy/h.png}
% \end{minipage}
\begin{minipage}[t]{0.5\textwidth}\setlength{\parindent}{0.45cm}  % V případě varianty č.2 jde odsud text do pravé části
\moveright 4.3cm \vbox{      %Varianta č. 1  ---> Jeden sloupec zarovnaný na střed	

\sloka
^{F{\color{white}\_}}Jede traktor, ^{As}je to Zetor,

^{B{\color{white}\_}\,\,\,\,}jede do hor ^{F{\color{white}\_}}orat brambor.

\sloka
Zemědělci brambor zasejí,

potom pohnojí, pak zas vyrejí.

Mládenec si kapsu namastí,

prachama zachrastí

na děvu povětrnou.

\refren

/: ^{F{\color{white}\_\_\_}As\,}Kriminalita, ^{F{\color{white}\_\_\_}As\,}kriminalita,
^{F{\color{white}\_\_\_}As\,}kriminalita, ^{B{\color{white}\_\_\_}}mládeže. :/

\sloka
Mládenec a děva spolu jdou

před novou hospodou, hrozně se radujou.

Nejdřív se tu spolu opijem,

družstevní prase zabijem

a pak si užijem.

\refren

\sloka
Mládenec se hrozně opije,

vepříka zabije a pak se vzapamatuje.

Svýho čimu ihned lituje,

za mříže putuje

i s děvou povětrnou.

\refren

\sloka
Jede traktor, je to Zetor,

jede do hor orat brambor.


}
\end{minipage}
\setcounter{Slokočet}{0}
\end{song}
