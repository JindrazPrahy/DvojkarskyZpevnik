\begin{song}{title=\predtitle \centering Marie \\\large Zuzana Navarová }  %% sem se napíše jméno songu a autor

\vspace*{.5cm}

\moveleft 0.8cm \vbox{
\begin{centerjustified}
\begin{varwidth}[t]{0.55\textwidth}\setlength{\parindent}{\pindent}  %Varianta č. 2 --> Dva sloupce
\vetsi
\sloka
Marie ^{Ami\z}má~se vracet, ta, co tu bydlí.

Marie má se vracet, tak postav ^{Dmi}židli.

Marie bílý racek, Marie ^{Ami\z}má~se vracet,

ty si tu ^{H7}dáváš dvacet, tak abys ^{E7}vstal.

\sloka
Marie má se vracet, co by ses divil.

Marie má se vracet, ta, cos ji mydlil.

Marie modrý ptáček, Marie moudivláček

kufříkem od natáček, ta, cos ji štval.

\refren
Marie

^{Ami\z}hoď~sem cihlu, má se vracet

trá ra ta ta.

\sloka
Marie má se vracet, píšou, že lehce.

Marie má se vracet, že už tě nechce.

Ta, co jí není dvacet, Marie má se vracet.

Ta, cos jí dal pár facek a pak s ní spal.

\refren

\sloka
Marie má se vracet, ta co tu bydlí.

Marie má se vracet, tak postav židli,

olej a těžký kola, vlak někam do Opola,

herbatka, jedna Cola, no tak se sbal.

\refren

\end{varwidth}\mezisloupci\begin{varwidth}[t]{0.55\textwidth}\setlength{\parindent}{\pindent}
\vspace*{.42cm}

\sloka
Marie ^{Dmi \z}holubice,

Marie ^{Ami \z}létavice,

Marie ^{H7 \z}blýskavice,

krasavice, ^{E7}ech -- Rosice, Pardubice.


\sloka = 1.

\sloka
Marie má se vracet, co by ses divil,

Marie má se vracet, ta, cos ji mydlil.

Marie modrý ptáček, Marie moudivláček.

Tak postav ^{Dmi}židli a ^{E7}sbal si fidli. ^{Ami}

\end{varwidth}

\end{centerjustified}
}

\setcounter{Slokočet}{0}
\end{song}
