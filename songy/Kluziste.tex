\begin{song}{title=\centering Kluziště \\\normalsize Karel Plíhal \vspace*{-0.3cm}}  %% sem se napíše jméno songu a autor
\moveright 4cm \vbox{      %Varianta č. 1  ---> Jeden sloupec zarovnaný na střed	

\sloka 
	^{C{\color{white}aaaa}}Strejček ^{Emi7/H}kovář ^{Ami7}chytil ^{C}kleště, 

	^{Fmaj7}uštíp' z ^{C}noční ^{\,\,\,Fmaj7\,\,G}oblohy
	
	jednu malou kapku deště, ta mu spadla pod nohy,
	
	nejdřív ale chytil slinu, tak šáh' kamsi pro pivo,
	
	pak přitáhl kovadlinu a obrovský kladivo.

\refren
	Zatím ^{C}tři bílé ^{Emi7/H}vrány ^{Ami7}pěkně za ^{C}sebou
	
	kolem ^{Fmaj7}jdou, někam ^{C}jdou, do ^{D7}rytmu se ^{G}kývaj,

	tyhle ^{C}tři bílé ^{Emi7/H}vrány ^{Ami7}pěkně za ^{C}sebou
	
	kolem ^{Fmaj7}jdou, někam ^{C}jdou, ^{Fmaj7}nedojdou, ^{C}nedojdou.


\sloka	
	Vydal z hrdla mocný pokřik ztichlým letním večerem
	
	pak tu kapku všude rozstřík' jedním mocným úderem,
	
	celej svět byl náhle v kapce a vysoko nad námi
	
	na obrovské mucholapce visí nebe s hvězdami.

\refren

\sloka
	Zpod víček mi vytrysk' pramen na zmačkané polštáře,
	
	kdosi mě vzal kolem ramen a políbil na tváře,
	
	kdesi v dálce rozmazaně strejda kovář odchází,
	
	do kalhot si čistí dlaně umazané od sazí. 


}
\setcounter{Slokočet}{0}

\begin{center}
	\vspace*{1.0in}
	\includegraphics[scale=0.5]{../taby/kluziste.PNG}

\end{center}

	
\end{song}


