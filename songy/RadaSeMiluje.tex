\begin{song}{title=\predtitle\centering Ráda se miluje \\\large Karel Plíhal  \vspace*{-0.3cm}}  %% sem se napíše jméno songu a autor
\begin{centerjustified}
\nejnejvetsi

\refren
^{Hmi}Ráda se miluje, ^{A{\z}}ráda ^{D\,}jí,

^{G{\z}}ráda si ^{F^{\#}mi\,}jenom tak ^*{Hmi}zpívá ,

^{Hmi\z}vrabci se na plotě ^{A\,{\z}D}hádají,

^*{G}ko lik ^{\,\,F^{\#}mi}že~času~jí ^{Hmi\z}zbývá.

\sloka
^{G}Než vítr dostrká k ^{D\z}útesu ^{G\z}tu~její ^{\z D\,\,F^{\#}mi}legrační~bárku~~~~

a ^{Hmi\z}Pámbu si ve svým ^{A\z D}notesu ^{G\z F^{\#}mi\:\:\:\:\:\:}udělá~jen~další ^{Hmi\z}čárku.

\refren

\sloka
Psáno je v nebeské režii, a to hned na první stránce,

že naše duše nás přežijí v jinačí tělesný schránce.

\refren

\sloka
Úplně na konci paseky, tam, kde se ozvěna tříští,

sedí šnek ve snacku pro šneky -- snad její podoba příští.


\refren

\end{centerjustified}
\setcounter{Slokočet}{0}
\end{song}

\begin{figure}[h]
\predtitle\centering
\includegraphics[width=3cm]{../Akordy/hm.png}
\includegraphics[width=3cm]{../Akordy/a.png}
\includegraphics[width=3cm]{../Akordy/d.png}
\includegraphics[width=3cm]{../Akordy/g.png}
\includegraphics[width=3cm]{../Akordy/fxm.png}
\end{figure}
