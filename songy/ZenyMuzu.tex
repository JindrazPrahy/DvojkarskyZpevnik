\begin{song}{title=\predtitle\centering Ženy mužů \\\large Epydemye \vspace*{-0.1cm}}  %% sem se napíše jméno songu a autor
\begin{centerjustified}
\nejvetsi

\textbf{Předehra:}
\writechord{C F C G C}

\begin{varwidth}[t]{0.48\textwidth}\setlength{\parindent}{\pindent}

\sloka
^{C\z}Svěřit se křídlům ^{F\z}falešným

a ^{C\z}silou ^*{\z G}nečisto u se vznést

a ^{C\z}najít pole ^{F\z}božích vín,

to ^{C\z}stojí ^{G\z}někdy ^{C\z}trest.


^{\z F}Čekají na šťastné ^{C \z}přistání,

^{\z Ami}umět tak směry ^{G\z}bouří vést,

co ^{Dmi\z}ještě ^{Ami\z}všechno ^{F\z}musí ^{Emi\z}snést

ženy ^{F\z}mužů ^*{G\z C}létavý ch.

\sloka
Ke skále denně přikován

a na barevných nitkách svět,

těžký vrchol je překonán,

do sbírky další květ.

Čekající dole na zemi,

umět tak pevná lana plést,

co ještě všechno musí snést

ženy mužů lezoucích.

\sloka
Oporu svou mít ve zbrani

a domů dálka celý svět,

jenom se zítra ubránit

a pak se vrátit zpět.

Čekající denně na návrat,

umět tak zahnat černou zvěst,

co ještě všechno musí snést

ženy mužů statečných.

\end{varwidth}\mezisloupci\begin{varwidth}[t]{0.53\textwidth}\setlength{\parindent}{\pindent}
\vspace*{0.42cm} % V případě varianty č.2 jde odsud text do pravé části

\sloka
Svůj svět mít doma v komoře,

ze snů z dětství stavět hrad,

točí se vláčky v oboře,

až večer cítit hlad.

Čekající doma v kuchyni,

umět tak nudu prostě smést,

co ještě všechno musí snést

ženy mužů domácích.

\sloka
Domů se po schodech potácet

a nadávat už i na sebe,

před zraky ženy se vytrácet,

zamčené postele.

Čekající s pláčem v předsíni,

umět tak znovu chlapa svést,

co ještě všechno musí snést\elipsa.\elipsa.\elipsa.

\phantom{.}


.\elipsa.\elipsa.\elipsa ^{C\z}ženy mužů ^{F\z}létavých,

ženy mužů ^{G\z}lezoucích,

^{C\z}ženy mužů ^{F\z}statečných,

^{C\z}ženy mužů ^*{G\z C}domácí ch.


\sloka
Čekající s pláčem v předsíni,

umět tak znovu chlapa svést.

Co ještě všechno musí snést

ženy mužů ztracených.

\end{varwidth}

\end{centerjustified}

\setcounter{Slokočet}{0}
\end{song}
