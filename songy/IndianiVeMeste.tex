\begin{song}{title=\predtitle\centering Indiáni ve městě \\\large Traband  \vspace*{-0.3cm}}  %% sem se napíše jméno songu a autor

\moveleft 0.5cm \vbox{

\begin{centerjustified}
\begin{varwidth}[t]{0.58\textwidth}\setlength{\parindent}{\pindent}  %Varianta č. 2 --> Dva sloupce
\vspace*{0.6cm}
\refren
^{Ami\z}Indiáni ve městě, indiáni ve městě, ^{\z G\,C}indiáni

^{C\z}Indiáni ^{Ami\z}ve~městě, ^{C\z}indiáni ^{Ami\z}ve~městě, ^{\z G\,F}indiáni

^{\z G\,Ami}Indiáni.~~~~

\sloka
Starý ^{Ami\z}muž~sedí u ohně a vypráví příběh

z dávných ^{G\z}dob, kdy se mladí chlapci ^{E7\z}vydali

na cestu

^{Ami\z}Za~světlem a za potravou, za dobrým lovem,

^{G\z}za~místem k životu, ^{E7\z}za~obzorem

ale ^{F\z}nikdo z nich z té cesty už nevrátil se ^{Ami}zpět

^{F\z}Všichni dávno odešli a stařec zbyl tu ^{Ami}sám. ^{G}


\refren

\sloka
Někde daleko a jinde sedí mladí muži znehybnělí

s tvářemi ozářenými obrazovkami...

Na kterých se míhají postavy a démoni,

co vedou boj o něčí život, o ztracenou duši.

Neříkají nic, jen slova, co nedávají smysl.

Posílají zprávy o ničem, odnikud a nikam.

Potom jedou v silných vozech zabíjet nudu.

A umírají sami - ani bojovníci, ani lovci.

\refren

\end{varwidth}\mezisloupci\begin{varwidth}[t]{0.41\textwidth}\setlength{\parindent}{\pindent}
\vspace*{0.60cm} % V případě varianty č.2 jde odsud text do pravé části

\sloka
^{D\z}A~dýmka míru stále ^{Ami}hoří

^{C\z}Voňavé byliny a ^{G\z}koření

Na nebi, na zemi i v moři

Velký duch bdí nad svým

stvořením

\sloka
Ne, ještě nejsem u konce, ještě

mám dost sil,

abych vylomil mříže z týhle

pozlacený klece

a vyšel se svou duší do lesů a do

skal

dál proti proudu ve špinavý řece.

A tam někde vysoko v horách

našel starce,

který hlídá oheň a vypráví

příběhy.

Přisedl bych k němu, pokorně

a tiše...

Mlčel bych a poslouchal...

Mlčel bych a poslouchal...

\refren ... ve mě jste ...
\end{varwidth}

\end{centerjustified}
}

\setcounter{Slokočet}{0}
\end{song}
