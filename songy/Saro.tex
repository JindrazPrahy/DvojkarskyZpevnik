\begin{song}{title=\predtitle\centering Sáro! \\\large Traband  \vspace*{-0.3cm}}  %% sem se napíše jméno songu a autor

\moveright -.5cm \vbox{
\begin{centerjustified}
\nejvetsi

\begin{varwidth}[t]{0.48\textwidth}\setlength{\parindent}{\pindent}  %Varianta č. 2 --> Dva sloupce
\sloka 
	^{Ami\,}Sáro, ^{Emi\,}Sáro, ^{F}v noci se mi ^{C\z}zdálo,

	že ^{F}tři andělé ^{C{\z}}Boží k nám ^{F{\z}}přišli na 

	^{G\z}oběd.

	^{Ami\,}Sáro, ^{Emi\,}Sáro, jak ^{F\,\,}moc a nebo ^{C\z}málo

	mi ^{F\z}chybí abych ^{C\z}tvojí duši ^{F\z}mohl 

	^{\,\,\,G}rozumět?

\sloka
	Sbor kajícných mnichů jde krajinou 

	v tichu

	a pro všechnu lidskou pýchu

	má jen přezíravý smích.

	A z prohraných válek se vojska 

	domů vrací,

	však zbraně stále burácí

	a bitva zuří v nich.

\sloka
	Vévoda v zámku čeká na balkóně,

	až přivedou mu koně

	a pak mává na pozdrav.

	A srdcová dáma má v každé ruce 

	růže.

	Tak snadno poplést může

	sto urozených hlav.

\sloka
	Královnin šašek s pusou od povidel

	sbírá zbytky jídel

	a myslí na útěk.

	A v podzemí skrytí slepí alchymisté

	už objevili jistě

	proti povinnosti lék.

\end{varwidth}\mezisloupci\begin{varwidth}[t]{0.48\textwidth}\setlength{\parindent}{\pindent}
\vspace*{0.405cm}  % V případě varianty č.2 jde odsud text do pravé části

\sloka
	Páv pod tvým oknem zpívá sotva 

	procit

	o tajemstvích noci

	ve tvých zahradách.

	A já -- potulný kejklíř, co svázali 

	mu ruce,

	teď hraju o tvé srdce

	a chci mít tě na dosah.

\sloka
	^{Ami\,}Sáro, ^{Emi\,}Sáro, ^{F\z}pomalu a ^{C\,\,}líně

	^{F\z}s~hlavou na tvém ^{C\z}klíně ^{F\z}chci se

	^*{{\z}G}probouz et.

	^{F{\z}}Sáro, Sáro, ^{C}Sáro, Sáro ^{F\,\,}rosa padá

	^{C\z}ráno

	a ^{F}v poledne už ^{C{\z}}možná ^{F{\z}}bude jiný

	^{G\z}svět.

	^{F{\z}}Sáro, ^{C{\z}}Sáro, ^{F{\z}}vstávej, milá ^{C{\z}}Sáro!

	^{F{\z}}Andělé k nám ^{Dmi}přišli na ^{\,\,Cmaj7}oběd.


\end{varwidth}

\end{centerjustified}
}
\setcounter{Slokočet}{0}
\end{song}
