\begin{song}{title=\predtitle \centering Antidepresivní rybička \\\large Vypsaná Fixa   \vspace*{-0.3cm}}  %% sem se napíše jméno songu a autor
\begin{centerjustified}
\nejnejvetsi
\begin{varwidth}[t]{0.48\textwidth}\setlength{\parindent}{0.45cm}  %Varianta č. 2 --> Dva sloupce
\sloka 
  ^{C}Ona má ^*{\z Emi}antidepresiv ní ^{F\z }rybičku

  ^*{C}vy tetovanou na 
  
  ^*{Emi}nejta jnějším ^{F\z }místě.

  Možná je pod srdcem 

  a možná trochu níž.

  Ona může všude,

  prostě tam, kam 
  
  si vymyslíš.

\refren
  A ona ^{C\z }plave 

  z orgánu ^{G}do orgánu.

  Žere ^{Ami\,\,}plevel,

  ^{D\,\,}kterej po ránu

  ^*{C}om otává mozek

  a ^*{G}ko tníky na konci 
  
  ^{\z Ami}pelesti.\phantom{xt}

  ^{D}Hu!


\end{varwidth}\mezisloupci \begin{varwidth}[t]{0.48\textwidth}\setlength{\parindent}{0.45cm}
\vspace*{0.465cm}  % V případě varianty č.2 jde odsud text do pravé části

\refren
  A ona plave 
  
  z modřiny do modřiny

  a vygumuje 

  je úplně všechny

  a tvoje vnitřní orgány 

  tolerují její rebelství.

  Hu!

\sloka
  Veze si antidepresivní rybičku

  městem co má velkou spoustu 
  
  pastí.

  Falešný městský strážník,

  bary a zloděj kol 
  
  a nebo prostě každý, 

  koho si ty 

  vymyslíš.

\refren
 /:  Ona je rebel, 
  
  kouří na zastávce,

  jí to chutná  

  v týhle válce 

  a její vnitřní orgány

  tolerují její rebelství. :/

\sloka 
  /: A potom přijde DJ PUNK,

  bude to nejlepší herec a další 

  a celej underground 

  i ryba pod monopolem. :/


\end{varwidth}
\end{centerjustified}
\setcounter{Slokočet}{0}
\end{song}

