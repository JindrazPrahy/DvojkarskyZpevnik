\begin{song}{title=\centering Antidepresivní rybička \\\normalsize Vypsaná Fixa   \vspace*{-0.3cm}}  %% sem se napíše jméno songu a autor
\moveright 1.5cm \vbox{      %Varianta č. 1  ---> Jeden sloupec zarovnaný na střed	

\begin{minipage}[t]{0.48\textwidth}\setlength{\parindent}{0.45cm}  %Varianta č. 2 --> Dva sloupce
\sloka 
  Ona má ^{C\,\,\,Emi}antidepresivní ^{F}rybičku

  ^{C}vytetovanou na ^{Emi}nejtajnějším ^{F}místě.

  Možná je pod srcdem 

  a možná trochu níž.

  Ona může všude

  prostě tam kam 
  
  si vymyslíš.

\sloka
  A ona ^{C}plave 

  z orgánu ^{G}do orgánu.

  Žere ^{Ami}plevel

  ^{D}kterej po ránu

  ^{C}omotává mozek

  a kotníky na konci ^{Ami}pelesti.

  ^{D}Hu!

\sloka
  A ona plave 
  
  z modřiny do modřiny

  a vygumuje 

  je úplně všechny

  a tvoje vnitřní orgány 

  tolerují její rebelství.

  Hu!


\end{minipage}\begin{minipage}[t]{0.48\textwidth}\setlength{\parindent}{0.45cm}\vspace*{0.55cm}  % V případě varianty č.2 jde odsud text do pravé části

\sloka
  Veze si antidepresivní rybičku

  městem co má velkou spoustu pastí.

  Falešný městský strážník,

  bary a zloděj kol 
  
  a nebo prostě každý 

  koho si ty 

  vymyslíš.


\sloka
 /:  Ona je rebel 
  
  kouří na zastávce,

  jí to chutná, 

  v týhle válce 

  a její vnitřní orgány

  tolerují její rebelství. :/

\sloka 
  /: A potom přijde DJ PUNK,

  bude to nejlepší herec a další 

  a celej underground 

  i ryba pod monopolem. :/



\end{minipage}
}
\setcounter{Slokočet}{0}
\end{song}

