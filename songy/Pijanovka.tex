\begin{song}{title=\predtitle\centering Pijánovka \\\large Tři sestry  \vspace*{-0.3cm}}  %% sem se napíše jméno songu a autor
\begin{centerjustified}

\sloka
^{D}Na highway číslo jedna ^*{A}vy jetejma kolejema

v ^*{G}pr achu na pět svíček brachu ^*{A}je zděj traky beze strachu.

^*{D}Mo žná potkáš cestou Láďu, ^{A\,\,}jak žene svůj dravej vůz,

^{G\z}svojí starou rychlou káru, ^*{A}kt erý říká autobus.

^{D}V Průhonicích vezme plnou, ^{A}na padesátým spláchne kolou

^{G \z}rybu, kterou snědl celou ^*{A}sm aženou jen trochu leklou.

\refren
/: ^*{D}Hi ghway číslo ^{Hmi}jedna a po ní jede ^*{A}ki ng.

^*{D}Lá ďa jede ^*{Hmi}autob usem ^{G}je fakt highway ^*{A}ki ng.:/

\sloka
U Humpolce v příkrym kopci často

bouraj politici, Láďa vždycky zapálí si

camelku jak dobrodruzi, u Křížů si

koupí žvejku, velkej stejk a pěknou holku.

Na cestě je chlapům smutno, zvlášť když

nemaj rychlou ruku. Každej řidič má svůj

příběh, někdo zdrhá před osudem,

někdo zdrhá před svou starou

nebo ztratil prachy s vírou.


\refren


\sloka
Láďa nosí ve svém srdci příběh černej

jako krtci, plnej zrady, krve, pěstí,

závisti a nenávisti. Jeho příběh je tak

krutej, že je ze všech nejkrutější,

že ho Láďa nosí v sobě a zemře s ním

ve svym hrobě. A tak končí tahle píseň

o Láďovi s autobusem. Jeho příběh mi

neznáme, a tak o něm nezpíváme.


\refren

\end{centerjustified}
\setcounter{Slokočet}{0}
\end{song}
