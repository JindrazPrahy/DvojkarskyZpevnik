\begin{song}{title=\predtitle\centering Strom kýve pahýly \\\large Divadlo Sklep \vspace*{-0.3cm}}  %% sem se napíše jméno songu a autor
\begin{centerjustified}

\sloka
	^{C\z}Když slunce ^{Dmi \z}zapadá, tak ^{C \z}moje ^{\z Fmaj7}nálada~~~~~ ^*{\z C\,}klesá, ^{Dmi\,\,C\,\,Fmaj7}

	^{C\z}strom kýve ^{Dmi\z}větvemi, ^{C}přítelem on je ^{Fmaj7}mi,~~~~ ^{\z C\,}plesá, ^{Dmi\,\,C\,\,Fmaj7}

	^{C}já však mám v duši žal, čert ^{Dmi}ví, kde se tam vzal,

	^{C}tepe,~~ ^{\z Dmi}tepe,~~~ ^{C}tepe,~~ ^{\z Fmaj7}tepe.~~~~

\refren
	^{C\z}Strom kýve ^{Dmi \z}pahýly, chtěl ^{C \z}bych jen na ^{\z Fmaj7}chvíli~~~~~~ tebe,

	^{C\z}strom kýve ^{Dmi \z}pahýly, chtěl ^{C \z}bych jen na ^{\z Fmaj7}chvíli~~~~~~ tebe,

	^{C\z}rosu mám v ^{Dmi \z}kanadách, v mých ^{C \z}černých ^{\z Fmaj7}kanadách~~~~ zebe,

	^{C \z}rosu mám v ^{Dmi \z}kanadách, v mých ^{C \z}černých ^{\z Fmaj7}kanadách~~~~ zebe.


\sloka
	Znám dobře kůru lip, té dal jsem kdysi slib mlčení,

	znám řeč, jíž mluví hřib, sám jako jedna z ryb jsem němý,

	však lesy, ty mám rád, tam cítím se vždycky mlád,

	vždycky líp, vždycky líp, vždycky líp, vždycky líp.



\refren

\end{centerjustified}
\setcounter{Slokočet}{0}
\end{song}
