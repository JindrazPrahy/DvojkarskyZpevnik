%%%%%%%%%%%%%%%%%%%%%%%%%%%%%%%%%%%%%%%%%%%%%%%%%%%%%
%			GENERÁTOR v. 18.09 						%
%%%%%%%%%%%%%%%%%%%%%%%%%%%%%%%%%%%%%%%%%%%%%%%%%%%%%
% Tento soubor slouží k jednotlivé zkusné kompilaci 
% písniček či k kompilaci samostatných pdf souborů 
% pomocí bashového skriptu kompilace.sh ve stejné 
% složce. Z důvodu hromadné kompilace musí poslední 
% dva řádky tohoto souboru zůstat nezměněny.
%%%%%%%%%%%%%%%%%%%%%%%%%%%%%%%%%%%%%%%%%%%%%%%%%%%%%
%			Jak kompilovat jednotlivé písně?        %
%%%%%%%%%%%%%%%%%%%%%%%%%%%%%%%%%%%%%%%%%%%%%%%%%%%%%
%	1. Do míst PÍSNIČKA upravte následnující řádek, 
%	   aby vypadal takto: 
%	   \input{../songy/[JMÉNO SOUBORU KOMPILOVNÉ PÍSNIČKY].tex}
%	2. Soubor písničky musí být přeložitelný a musí 
%	   se nacházet ve složce ../songy.
%%%%%%%%%%%%%%%%%%%%%%%%%%%%%%%%%%%%%%%%%%%%%%%%%%%%%
%			Jak psát soubory songů?                 %
%%%%%%%%%%%%%%%%%%%%%%%%%%%%%%%%%%%%%%%%%%%%%%%%%%%%%
%	1. Více návodu je k tomuto napsáno v souboru 
%      ../songy/00Songtemplate. 
%%%%%%%%%%%%%%%%%%%%%%%%%%%%%%%%%%%%%%%%%%%%%%%%%%%%%
%			Jak kompilovat celý zpěvník?			%
%%%%%%%%%%%%%%%%%%%%%%%%%%%%%%%%%%%%%%%%%%%%%%%%%%%%%
%	1. Více návodu je k tomuto napsáno v souboru
%	   ../Cely_zpevnik/zpevnik.tex.
%%%%%%%%%%%%%%%%%%%%%%%%%%%%%%%%%%%%%%%%%%%%%%%%%%%%%

% Hlavička START
\documentclass[openany,12pt]{memoir}
\usepackage[utf8]{inputenc} 
\usepackage[czech]{babel}
\usepackage[T1]{fontenc}
\usepackage[top=1.5cm, bottom=2cm, left=2cm, right=2cm]{geometry}  % --> NASTAVENÍ OKRAJŮ
\usepackage{fancyhdr}
\usepackage{graphicx}
\usepackage{xwatermark}
\usepackage{xcolor}
\usepackage{changepage}
\usepackage{pdfpages}
\usepackage{lettrine}
\usepackage{indentfirst}  %Důležité pro formátování
\usepackage{multicol}

%%%%%%%%%%%%%%%%%%%%%%%%%%%%%%%%%%%%%%
%  FONT                              %
%%%%%%%%%%%%%%%%%%%%%%%%%%%%%%%%%%%%%%
\usepackage[charter]{mathdesign}



%%%%%% Package na zpěvník
\usepackage[full]{leadsheets}%http://mirrors.nic.cz/tex-archive/macros/latex/contrib/leadsheets/leadsheets_en.pdf   --> dokumentace	
\definesongtitletemplate{empty}{} 
\setchords{
format = \bfseries,% \normalsize , %tučné akordy
minor = {mi},% 
input-notation = {german},%
output-notation = {german}%
}
\definesongtitletemplate{empty}{} 

\newlength{\drop}
\newwatermark[pages=-,color=red!50,angle=0,scale=2, xpos=0,ypos=0]{\includegraphics[width=5cm]{obr/pozadi2.jpg}} %--> dvojka na pozadí


%%%%%%%%%%%%%%%%%%%%%%%%%%%%%%%%%%%%%%%%%%%%%%%%%%
%		 Vlastní příkazy
\newcounter{Slokočet}   %Automatické číslování slok
\newcommand{\mezera}{\textcolor{white}{\,}\\}   %Horizontální odsazení slok
\newcommand{\stred}{5.2cm}   %%% Na zarovnání slok doprostřed, pozn. automatičtější zarovnávání na střed nejde
\newcommand{\carka}{,\distanc}
\newcommand{\m}[1]{\color{white}{#1}}  %Pro akordy
\newcommand{\ap}{'}	%Pro apostrof
\newcommand{\elipsa}{\kern\fontdimen3\font} %Příkaz pro lepší zacházení s výpustkami (=...); je to vpodstatě jen mezera mezi tečkama výpustky
\newcommand{\pindent}{17.62482 pt} %Správná velikost \parindentu u layoutu se dvěma minipageama
\newcommand{\normalni}{\normalsize}
\newcommand{\velky}{\large}
\newcommand{\vetsi}{\Large}

%%% Stará definice sloky spoléhající na indenty
%\newlength{\pismeno}
%\settowidth{\pismeno}{x} %Tohle není moc ideální velikost, ale funguje
%\newif\ifslokavelka
%\slokavelkafalse
%\newcommand{\sloka}{
%\ifnum \value{Slokočet}>8  %Pokud je sloka dvouciferná
%\mezera \noindent \addtocounter{Slokočet}{1} \hspace*{-\pismeno}\arabic{Slokočet}.
%\else %Pokud jen jednociferná
%\mezera \noindent \addtocounter{Slokočet}{1} \arabic{Slokočet}. 
%\fi
%} 	%sloka, která se automaticky čísluje

\newcommand{\distanc}{\:} %Vzdálenost čísla sloky před slokou
\newlength{\delkaargumentu}
%%% Sloka s automatickým číslováním
\newcommand{\sloka}{%
\addtocounter{Slokočet}{1}% Zvýší se o 1 počet slok
\mezera%  Sloka se odsadí vertikálně
\settowidth{\delkaargumentu}{\arabic{Slokočet}.\distanc}% Zde se určí délka odsazení 
\hspace*{-\delkaargumentu}%
\arabic{Slokočet}.\distanc%
\ignorespaces%  Aby nevznikaly zbytečné mezery
}


%%% Sloka s vlastním argumentem
\newcommand{\ssloka}[1]{%     
\settowidth{\delkaargumentu}{#1\distanc}%
\mezera%
\hspace*{-\delkaargumentu}%
#1\distanc%
\ignorespaces%
}  

%%% Refrén
\newcommand{\refren}[1][0]{%  Nepovinný argument sděluje, kolikátý refrén toto je, bez argumentu se vytiskne pouze refrén
\ifnum #1>0 %Pokud nepovinný argument existuje
\mezera%
\settowidth{\delkaargumentu}{\textbf{R$_{\text{#1}}$:}\distanc}%
\hspace*{-\delkaargumentu}%
\textbf{R$_{\text{#1}}$:}\distanc%
\ignorespaces%
\else %Pokud nepovinný argument neexistuje
\mezera%
\settowidth{\delkaargumentu}{\textbf{R:}\distanc}%
\hspace*{-\delkaargumentu}%
\textbf{R:}\distanc%
\ignorespaces%
\fi
}

\newcommand{\predehra}{\ssloka{\textbf{Předehra:}}}

\addto\captionsczech{\renewcommand{\contentsname}{Seznam písní}}

%%%%%%%%%%%%%%%%%%%%%%%%%%%%%%%%%%%%
%    FORMÁTOVÁNÍ                   %
%%%%%%%%%%%%%%%%%%%%%%%%%%%%%%%%%%%%

%%% Vlevo zarovnaný text s blokem zarovnaným na střed
\usepackage{varwidth}% http://ctan.org/pkg/varwidth
\newenvironment{centerjustified}{%
  \begin{center} % so the minipage is centered
  \begin{varwidth}[t]{\textwidth}	
  \raggedright % so the minipage's text is left justified
  \setlength{\parindent}{\pindent}
}{%
  \end{varwidth}
  \end{center}
}
%Hlavnička END
\begin{document}
\sffamily %Sans-serif pro lepší čitelnost

\pagestyle{empty}
\newgeometry{top=1.5cm, bottom = 1cm, left = 2cm, right = 2cm}
	
\newpage

%Pro kompilaci je důležitý, aby ten \input a \end{document} zůstali na posledních
%dvou řádcích!!!!

\newpage
% PÍSNIČKA
%%%%%%%%%%%%%%%%%%%%%%%%%%%%%%%%%%%%%%%%%%%%%%%%%%%%%
%			ŠABLONA PÍSNIČEK v. 18.09               %
%%%%%%%%%%%%%%%%%%%%%%%%%%%%%%%%%%%%%%%%%%%%%%%%%%%%%
% Tento soubor slouží jako (naučná) šablona, pomocí 
% které lze vytvářet zdrojové soubory k jednotlivým 
% písním.
%%%%%%%%%%%%%%%%%%%%%%%%%%%%%%%%%%%%%%%%%%%%%%%%%%%%%
%			Jak psát soubory songů?                 %
%%%%%%%%%%%%%%%%%%%%%%%%%%%%%%%%%%%%%%%%%%%%%%%%%%%%%
%	1. Text písně se začíná psát na místě START 
%	   a končí na místě END. Zbylý text ignorujte.
%	2. Jak bude vypadat pdf písně zjistíte po tom, 
%	   co soubor zkompilujete pomocí souboru   
%      ../Generator/generator. 
%	3. Při psaní dodržujte následující TeX pravidla:
%	 a) Nový řádek napíšete pomocí dvou odsazení 
%	    tedy dvou enterů.
%	 b) Nová sloka se píší pomocí \sloka a odsazení.
%		Refrén se píše jako \refren, v případě více 
%		refrénů \refren[č. refrénu].
%	 c) Akordy se píšou tak, že napíšete před slovo,
%	    kde chcete mít akord (bez mezery):
%		^{AKORD1\,AKORD2...}.
%	4. Pokud chcete ušetřit tvůrcům práci, tak 
%	   si přečtěte další poučný soubor o typografii 
%	   ../../Typo_pravidla.txt.
%	5. Akordy stačí psát jen do první sloky, když 
%	   se nezmění -- kytaristé to zvládnou
%	7. Název písně pište na místo [NÁZEV] a autora 
%	   pište na místo [AUTOR] 
%	7. Jak psát věci na české klávesnici:
%	   \ = alt gr + q; [/] = alt gr f/g; 
%      {/} = alt gr + b/n; ^ = alt gr + 3 , cokoliv
%%%%%%%%%%%%%%%%%%%%%%%%%%%%%%%%%%%%%%%%%%%%%%%%%%%%%
%			Jak kompilovat jednotlivé písně?        %
%%%%%%%%%%%%%%%%%%%%%%%%%%%%%%%%%%%%%%%%%%%%%%%%%%%%%
%	1. Více návodu je k tomuto napsáno v souboru 
%      ../Generator/generator. 
%%%%%%%%%%%%%%%%%%%%%%%%%%%%%%%%%%%%%%%%%%%%%%%%%%%%%
%			Jak kompilovat celý zpěvník?			%
%%%%%%%%%%%%%%%%%%%%%%%%%%%%%%%%%%%%%%%%%%%%%%%%%%%%%
%	1. Více návodu je k tomuto napsáno v souboru
%	   ../Cely_zpevnik/zpevnik.tex.
%%%%%%%%%%%%%%%%%%%%%%%%%%%%%%%%%%%%%%%%%%%%%%%%%%%%%
\begin{song}{title=\predtitle \centering [NÁZEV] \\\large [AUTOR] }  %% sem se napíše jméno songu a autor

\vspace*{.5cm}

\begin{centerjustified}
\vetsi
\sloka  %1
Sloka ^{E}nana na nanananana

^{Ami}Nové odsazení ^{D}lalala lalalallaa

\velky
\refren 

/: Refrén v ^{Ami}repetici

^{D}jak chytlavé\elipsa.\elipsa.\elipsa. :/

\vetsi
\sloka  %1
Sloka ^{E}nana na nanananana

^{Ami}Nové odsazení ^{D}lalala lalalallaa

\refren 

/: Refrén v ^{Ami}repetici

^{D}jak chytlavé\elipsa.\elipsa.\elipsa. :/


\end{centerjustified}
\setcounter{Slokočet}{0}
\end{song}

\end{document}