%\documentclass[../main.tex]{subfiles}

\begin{song}{title=\centering Slavíci z Madridu \\\normalsize Waldemar Matuška \vspace*{-0.3cm}}  %% sem se napíše jméno songu a autor
\moveright 4.5cm \vbox{      %Varianta č. 1  ---> Jeden sloupec zarovnaný na střed	

\refren[1]
	Na\elipsa\dots\,na na\elipsa\dots

\sloka
	^{Ami\,}Nebe je modrý a ^{E{\color{white}\_\_}}zlatý, bílá sluneční ^{Ami\,\,}záře,

	horko a sváteční ^{E{\color{white}\_}}šaty, vřava a zpocený ^{Ami\,\,\,}tváře. 

	Vím, co se bude ^{E\,\,}dít, býk už se v ohradě ^{Ami\,{\color{white}\_}}vzpíná,

	kdo chce, ten může ^{E\,\,}jít, já si dám sklenici ^{Ami\,\,}vína.

\refren[2]
	^{Dmi\,}Žízeň je veliká, ^{Ami\,}život mi utíká,

	^{E{\color{white}\_\_}}nechte mě příjemně ^{Ami\,\,}snít,

	^{Dmi}ve stínu pod fíky ^{Ami{\color{white}\_\_\_}}poslouchat slavíky,

	^{E{\color{white}\_\_}}zpívat si s nima a ^{Ami}pít.

\sloka
	Ženy jsou krásný a cudný, mnohá se ve mně zhlídla,

	oči jako dvě studny, vlasy jak havraní křídla.

	Dobře vím, co znamená pád do nástrah dívčího klína,

	někdo má pletky rád, já si dám sklenici vína.

\refren[2]

\sloka
	Nebe je modrý a zlatý, ženy krásný a cudný,

	mantily, sváteční šaty, oči jako dvě studny.

	Zmoudřel jsem stranou od lidí, jsem jak ta zahrada stinná,

	kdo chce, ať mi závidí, já si dám sklenici vína.
	
\refren[2]

\refren[1]

}
\setcounter{Slokočet}{0}

\end{song}
\begin{figure}[h]
\centering
\includegraphics[width=3cm]{../Akordy/am}
\includegraphics[width=3cm]{../Akordy/e}
\includegraphics[width=3cm]{../Akordy/dm}
\end{figure}

