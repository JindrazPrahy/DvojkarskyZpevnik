%%%%%%%%%%%%%%%%%%%%%%%%%%%%%%%%%%%%%%%%%%%%%%%%%%%%%
%			ŠABLONA PÍSNIČEK v. 18.09               %
%%%%%%%%%%%%%%%%%%%%%%%%%%%%%%%%%%%%%%%%%%%%%%%%%%%%%
% Tento soubor slouží jako (naučná) šablona, pomocí 
% které lze vytvářet zdrojové soubory k jednotlivým 
% písním.
%%%%%%%%%%%%%%%%%%%%%%%%%%%%%%%%%%%%%%%%%%%%%%%%%%%%%
%			Jak psát soubory songů?                 %
%%%%%%%%%%%%%%%%%%%%%%%%%%%%%%%%%%%%%%%%%%%%%%%%%%%%%
%	1. Text písně se začíná psát na místě START 
%	   a končí na místě END. Zbylý text ignorujte.
%	2. Jak bude vypadat pdf písně zjistíte po tom, 
%	   co soubor zkompilujete pomocí souboru   
%      ../Generator/generator. 
%	3. Při psaní dodržujte následující TeX pravidla:
%	 a) Nový řádek napíšete pomocí dvou odsazení 
%	    tedy dvou enterů.
%	 b) Nová sloka se píší pomocí \sloka a odsazení.
%		Refrén se píše jako \refren, v případě více 
%		refrénů \refren[č. refrénu].
%	 c) Akordy se píšou tak, že napíšete před slovo,
%	    kde chcete mít akord (bez mezery):
%		^{AKORD1\,AKORD2...}.
%	4. Pokud chcete ušetřit tvůrcům práci, tak 
%	   si přečtěte další poučný soubor o typografii 
%	   ../../Typo_pravidla.txt.
%	5. Akordy stačí psát jen do první sloky, když 
%	   se nezmění -- kytaristé to zvládnou
%	7. Název písně pište na místo [NÁZEV] a autora 
%	   pište na místo [AUTOR] 
%	7. Jak psát věci na české klávesnici:
%	   \ = alt gr + q; [/] = alt gr f/g; 
%      {/} = alt gr + b/n; ^ = alt gr + 3 , cokoliv
%%%%%%%%%%%%%%%%%%%%%%%%%%%%%%%%%%%%%%%%%%%%%%%%%%%%%
%			Jak kompilovat jednotlivé písně?        %
%%%%%%%%%%%%%%%%%%%%%%%%%%%%%%%%%%%%%%%%%%%%%%%%%%%%%
%	1. Více návodu je k tomuto napsáno v souboru 
%      ../Generator/generator. 
%%%%%%%%%%%%%%%%%%%%%%%%%%%%%%%%%%%%%%%%%%%%%%%%%%%%%
%			Jak kompilovat celý zpěvník?			%
%%%%%%%%%%%%%%%%%%%%%%%%%%%%%%%%%%%%%%%%%%%%%%%%%%%%%
%	1. Více návodu je k tomuto napsáno v souboru
%	   ../Cely_zpevnik/zpevnik.tex.
%%%%%%%%%%%%%%%%%%%%%%%%%%%%%%%%%%%%%%%%%%%%%%%%%%%%%
\begin{song}{title=\predtitle \centering Drobná paralela \\\large Chinaski }  %% sem se napíše jméno songu a autor

\vspace*{.5cm}

\begin{centerjustified}
\vetsi
\sloka
Ta ^{C\z}stará ^{G\z}dobrá hra je ^{D\z}okoukaná.

Nediv se brácho, kdekdo ji zná.

Přestaň se ptát, bylo nebylo líp.

Včera je včera, bohužel bohudík.

\refren
^{C}Nic není jako ^{G}dřív, ^{D}nic není jak ^{Emi}bejvávalo.

Nic není jako dřív, to se nám to mívávalo.

Nic není jako dřív, ačkoliv máš všechno co si vždycky chtěla

Nic není jako dřív, ačkoliv drobná paralela by tu byla.

\sloka
Snad nevěříš na tajný znamení.

Všechno to harampádí -- balábile -- mámení.

Vážení platící, jak všeobecně ví se,

včera i dneska, stále ta samá píseň.

\refren

\sloka
^*{C}Pro mlouvám k vám ^{G}ústy múzy,

^*{D}vzý vám tón a ^{Emi\z}lehkou chůzi,

^*{C}vzý vám zítřek ^*{G}nen adálý,

^*{D}odp louvám a ^{Emi\z}mizím.

\refren[2]
Nic není jako dřív, nic není jak bejvávalo.

Nic není jako dřív, jó, to se nám to dlouze kouřívalo.

Bohužel bohudík je s námi, ta nenahmatatelná intimita těla.

Nic není jako dřív, jen fámy, bla-bla-bla-bla et cetera.

Nic není jako dřív, nic není jak bejvávalo,

bohužel bohudík, co myslíš ségra, je to hodně nebo málo?

\end{centerjustified}
\setcounter{Slokočet}{0}
\end{song}
