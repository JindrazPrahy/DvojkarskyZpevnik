%%%%%%%%%%%%%%%%%%%%%%%%%%%%%%%%%%%%%%%%%%%%%%%%%%%%%
%			ŠABLONA PÍSNIČEK v. 18.09               %
%%%%%%%%%%%%%%%%%%%%%%%%%%%%%%%%%%%%%%%%%%%%%%%%%%%%%
% Tento soubor slouží jako (naučná) šablona, pomocí 
% které lze vytvářet zdrojové soubory k jednotlivým 
% písním.
%%%%%%%%%%%%%%%%%%%%%%%%%%%%%%%%%%%%%%%%%%%%%%%%%%%%%
%			Jak psát soubory songů?                 %
%%%%%%%%%%%%%%%%%%%%%%%%%%%%%%%%%%%%%%%%%%%%%%%%%%%%%
%	1. Text písně se začíná psát na místě START 
%	   a končí na místě END. Zbylý text ignorujte.
%	2. Jak bude vypadat pdf písně zjistíte po tom, 
%	   co soubor zkompilujete pomocí souboru   
%      ../Generator/generator. 
%	3. Při psaní dodržujte následující TeX pravidla:
%	 a) Nový řádek napíšete pomocí dvou odsazení 
%	    tedy dvou enterů.
%	 b) Nová sloka se píší pomocí \sloka a odsazení.
%		Refrén se píše jako \refren, v případě více 
%		refrénů \refren[č. refrénu].
%	 c) Akordy se píšou tak, že napíšete před slovo,
%	    kde chcete mít akord (bez mezery):
%		^{AKORD1\,AKORD2...}.
%	4. Pokud chcete ušetřit tvůrcům práci, tak 
%	   si přečtěte další poučný soubor o typografii 
%	   ../../Typo_pravidla.txt.
%	5. Akordy stačí psát jen do první sloky, když 
%	   se nezmění -- kytaristé to zvládnou
%	7. Název písně pište na místo [NÁZEV] a autora 
%	   pište na místo [AUTOR] 
%	7. Jak psát věci na české klávesnici:
%	   \ = alt gr + q; [/] = alt gr f/g; 
%      {/} = alt gr + b/n; ^ = alt gr + 3 , cokoliv
%%%%%%%%%%%%%%%%%%%%%%%%%%%%%%%%%%%%%%%%%%%%%%%%%%%%%
%			Jak kompilovat jednotlivé písně?        %
%%%%%%%%%%%%%%%%%%%%%%%%%%%%%%%%%%%%%%%%%%%%%%%%%%%%%
%	1. Více návodu je k tomuto napsáno v souboru 
%      ../Generator/generator. 
%%%%%%%%%%%%%%%%%%%%%%%%%%%%%%%%%%%%%%%%%%%%%%%%%%%%%
%			Jak kompilovat celý zpěvník?			%
%%%%%%%%%%%%%%%%%%%%%%%%%%%%%%%%%%%%%%%%%%%%%%%%%%%%%
%	1. Více návodu je k tomuto napsáno v souboru
%	   ../Cely_zpevnik/zpevnik.tex.
%%%%%%%%%%%%%%%%%%%%%%%%%%%%%%%%%%%%%%%%%%%%%%%%%%%%%
\begin{song}{title=\predtitle \centering Marie \\\large Zuzana Navarová }  %% sem se napíše jméno songu a autor

\vspace*{.5cm}

\begin{centerjustified}
\begin{varwidth}[t]{0.48\textwidth}\setlength{\parindent}{\pindent}  %Varianta č. 2 --> Dva sloupce
\vetsi
\sloka
Marie ^{Ami}má se vracet, ta, co tu bydlí.

Marie má se vracet, tak postav ^{Dmi}židli.

Marie bílý racek, Marie ^{Ami}má se

vracet,

ty si tu ^{H7}dáváš dvacet, tak abys ^{E7}vstal.

\sloka
Marie má se vracet, co by ses divil.

Marie má se vracet, ta, cos ji mydlil.

Marie modrý ptáček, Marie

moudivláček

kufříkem od natáček, ta, cos ji štval.

\refren
Marie

^{Ami\z}hoď~sem cihlu, má se vracet

trá ra ta ta

\sloka
Marie má se vracet, píšou, že lehce.

Marie má se vracet, že už tě nechce.

Ta, co jí není dvacet, Marie má se

vracet.

Ta, cos jí dal pár facek a pak s ní

spal.

\refren

\end{varwidth}\mezisloupci\begin{varwidth}[t]{0.5\textwidth}\setlength{\parindent}{\pindent}
\vspace*{.42cm}

\sloka
Marie má se vracet, ta co tu bydlí.

Marie má se vracet, tak postav židli,

olej a těžký kola, vlak někam do Opola,

herbatka, jedna Cola, no tak se sbal.

\refren

\sloka
Marie ^{Dmi}holubice,

Marie ^{Ami}létavice,

Marie ^{H7}blýskavice,

krasavice, ^{E7}ech -- Rosice, Pardubice.


\sloka = 1.

\sloka
Marie má se vracet, co by ses divil,

Marie má se vracet, ta, cos ji mydlil.

Marie modrý ptáček, Marie

moudivláček.

Tak postav ^{Dmi}židli a ^{E7}sbal si fidli ^{Ami}

\end{varwidth}

\end{centerjustified}
\setcounter{Slokočet}{0}
\end{song}
