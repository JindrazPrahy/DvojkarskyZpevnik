\documentclass[openany,12pt]{memoir}
\usepackage[utf8]{inputenc}
\usepackage[czech]{babel}
\usepackage[T1]{fontenc}
\usepackage[top=1.5cm, bottom=2cm, left=2cm, right=2cm]{geometry}  % --> NASTAVENÍ OKRAJŮ
\usepackage{fancyhdr}
\usepackage{graphicx}
\usepackage{xwatermark}
\usepackage{xcolor}
\usepackage{changepage}
\usepackage{pdfpages}
\usepackage{lettrine}
\usepackage{indentfirst}  %Důležité pro formátování
\usepackage[pages=some]{background}
\usepackage{wrapfig}

%%%%%%%%%%%%%%%%%%%%%%%%%%%%%%%%%%%%%%
%  FONT                              %
%%%%%%%%%%%%%%%%%%%%%%%%%%%%%%%%%%%%%%
\usepackage{amssymb}
\usepackage{tgschola}

%%%%%%%%%%%%%%%%%%%%%%%%%%%%%%%%%%%%%%
%  Obrázky v textu                   %
%%%%%%%%%%%%%%%%%%%%%%%%%%%%%%%%%%%%%%
\usepackage{tikz}
%\tikz[remember picture,overlay] \node[opacity=0.3,inner sep=0pt] at (current page.center){\includegraphics[width=\paperwidth,height=\paperheight]{example-image}};
%Tímto příkazem se na následující stránku vloží pozadí. Pokud pozadí uděláme
%tak, aby bylo velikosti a4, bylo prázdné až na malůvku, můžeme takto vkládat
%obrázky.
%Pozn.: kompilovat se musí dvakrát


%%%%%% Package na zpěvník
\usepackage[full]{leadsheets}%http://mirrors.nic.cz/tex-archive/macros/latex/contrib/leadsheets/leadsheets_en.pdf   --> dokumentace	
\definesongtitletemplate{empty}{} 
\setchords{
format = \bfseries \sffamily,   %tučné akordy
minor = {mi},% 
input-notation = {german},%
output-notation = {german}%
}
\definesongtitletemplate{empty}{} 


\newlength{\drop}
% VODOZNAK
\newwatermark[pages=3-,color=red!50,angle=0,scale=2, xpos=0,ypos=0]{\includegraphics[width=5cm]{obr/pozadi2.jpg}} %--> dvojka na pozadí


%%%%%%%%%%%%%%%%%%%%%%%%%%%%%%%%%%%%%%%%%%%%%%%%%%
%		 Vlastní příkazy
\newcommand{\defaulttabscale}{0.87}
\newcommand{\defaultfretscale}{1.7}

\newcounter{Slokočet}   %Automatické číslování slok
\newcommand{\mezera}{
\phantom{.}

}   %Horizontální odsazení slok (poněkud blbě zadefinovaný, ale jinak se formát rozbije jako wtf prostě)
\newcommand{\stred}{5.2cm}   %%% Na zarovnání slok doprostřed, pozn. automatičtější zarovnávání na střed nejde
\newcommand{\carka}{,\:}
\newcommand{\m}[1]{\color{white}{#1}}  %Pro akordy
\newcommand{\ap}{'}	%Pro apostrof
\newcommand{\elipsa}{\kern\fontdimen3\font} %Příkaz pro lepší zacházení s výpustkami (=...); je to vpodstatě jen mezera mezi tečkama výpustky
\newcommand{\pindent}{17.62482 pt} %Správná velikost \parindentu u layoutu se dvěma minipageama
\newcommand{\predtitle}{\huge}
\newcommand{\mezisloupci}{\phantom{TT}} %Místo mezi dvěma sloupci na jedné stránce
\newcommand{\z}{\hspace*{\fill}\null}

%%% Možné velikosti písem 
\newcommand{\normalni}{\normalsize}
\newcommand{\velky}{\fontsize{14.4}{15}\selectfont}
\newcommand{\vetsi}{\fontsize{15}{16}\selectfont}
\newcommand{\nejvetsi}{\fontsize{16}{17}\selectfont}
\newcommand{\nejnejvetsi}{\fontsize{17}{19}\selectfont}

%%% Stará definice sloky spoléhající na indenty
%\newlength{\pismeno}
%\settowidth{\pismeno}{x} %Tohle není moc ideální velikost, ale funguje
%\newif\ifslokavelka
%\slokavelkafalse
%\newcommand{\sloka}{
%\ifnum \value{Slokočet}>8  %Pokud je sloka dvouciferná
%\mezera \noindent \addtocounter{Slokočet}{1} \hspace*{-\pismeno}\arabic{Slokočet}.
%\else %Pokud jen jednociferná
%\mezera \noindent \addtocounter{Slokočet}{1} \arabic{Slokočet}. 
%\fi
%} 	%sloka, která se automaticky čísluje

\newcommand{\distanc}{\:}  %Vzdálenost čísla sloky před slokou
\newlength{\delkaargumentu}
%%% Sloka s automatickým číslováním
\newcommand{\sloka}{%
\addtocounter{Slokočet}{1}% Zvýší se o 1 počet slok
\mezera%  Sloka se odsadí vertikálně
\settowidth{\delkaargumentu}{\arabic{Slokočet}.\distanc}% Zde se určí délka odsazení 
\hspace*{-\delkaargumentu}%
\arabic{Slokočet}.\distanc%
\ignorespaces% Aby nevznikaly zbytečné mezery
}


%%% Sloka s vlastním argumentem
\newcommand{\ssloka}[1]{%     
\settowidth{\delkaargumentu}{#1\distanc}
\mezera%
\hspace*{-\delkaargumentu}%
#1\distanc%
\ignorespaces%
}  

%%% Refrén
\newcommand{\refren}[1][0]{%  Nepovinný argument sděluje, kolikátý refrén toto je, bez argumentu se vytiskne pouze refrén
\ifnum #1>0 %Pokud nepovinný argument existuje
\mezera%
\settowidth{\delkaargumentu}{\textbf{R$_{\text{#1}}$:}\distanc}%
\hspace*{-\delkaargumentu}%
\textbf{R$_{\text{#1}}$:}\distanc%
\ignorespaces%
\else %Pokud nepovinný argument neexistuje
\mezera%
\settowidth{\delkaargumentu}{\textbf{R:}\distanc}%
\hspace*{-\delkaargumentu}%
\textbf{R:}\distanc%
\ignorespaces%
\fi
}

\newcommand{\predehra}{\ssloka{\textbf{Předehra:}}}

%%% Capo
\newcommand{\kapodastr}[1]{
\textit{Capo \text{#1}}
}

\newcommand{\includefret}[1]{\includegraphics[scale=\defaultfretscale]{../Akordy/msc/#1.pdf}}

\addto\captionsczech{\renewcommand{\contentsname}{Seznam písní}}

%%%%%%%%%%%%%%%%%%%%%%%%%%%%%%%%%%%%
%    FORMÁTOVÁNÍ                   %
%%%%%%%%%%%%%%%%%%%%%%%%%%%%%%%%%%%%

%%% Vlevo zarovnaný text s blokem zarovnaným na střed
\usepackage{varwidth}% http://ctan.org/pkg/varwidth
\newenvironment{centerjustified}{%
  \begin{center} % so the minipage is centered
  \begin{varwidth}[t]{\textwidth}	
  \raggedright % so the minipage's text is left justified
  \setlength{\parindent}{\pindent}
}{%
  \end{varwidth}
  \end{center}
}

%%% Pozadí
\newcommand{\pozadi}[1]{
\backgroundsetup{
scale=1,
angle=0,
contents={%
  \includegraphics[width=\paperwidth,height=\paperheight]{#1}
  }%
}
\BgThispage
}

%%% Libovolné posouvání elementů (hlavně se používá na akordy),
%%% přičemž element má dimenze (0,0), takž neovlivní předchozí formátování.
\makeatletter
\newcommand*{\shifttext}[2]{%
  \settowidth{\@tempdima}{#2}%
  \makebox[\@tempdima]{\hspace*{#1}#2}%
}
\makeatother

%\cheatbox{X}{Y}{Element}
\newcommand{\cheatbox}[3]{%
\shifttext{#1}{\raisebox{#2}[0pt][0pt]{%
#3%
}}%
}


\usepackage{hyperref} %Musí být načteno jako poslední package
\begin{document}
\sffamily %Sans-serif font vypadá lépe
\velky   %Minimální čitelná velikost


\renewcommand{\abstractname}{\vspace{-\baselineskip}} %Aby nad úvodním textem nebyl nápis ,,Abstract''


%Kompilovat dvakrát, aby se updatnula TOC

\setcounter{page}{0}
\pagestyle{empty}
%%%%%%%%%%%% PŘEBAL1 %%%%%%%%%%%%%%%%%%
\includepdf[width=\paperwidth]{obr/Prebal_2022.pdf}
\newpage
\phantom{prázdná strana}
\newpage

%%%%%%%%%%%% Úvod %%%%%%%%%%%%%%%%%%
% \vspace*{4\baselineskip}
\begin{abstract}
\normalsize
\sffamily
% PŘEDMLUVA
\lettrine{D}{}ržíš v rukou další z řady oddílových zpěvníků \textbf{Pražské Dvojky} a \textbf{Sluníček}.
Obsah zpěvníku tvoří jak historické Dvojkařské písně z minulého století
tak i jen pár let existující aktuální hity, na něž jsme museli z odposlechu
skládat akordy a noty.

Příběh oddílového zpěvníku sahá až do písní minulých generací Dvojkařů: do
zpěvníku 2004 od Aleše a Sumce, do roverského zpěvníku 2012 od Krdlíka,
do~\uv{Old--School} zpěvníku 2015 od Sumce a do minulých verzí tohoto zpěvníku
2018 a 2020. Tyto zpěvníky jsou však dnes k~nalezení pouze v našich srdcích.

Do Dvojky a do Sluníček začaly přicházet nové generace kluků a holek.
Časem byly objevovnány nové i starší hudební klenoty.
Mnoho písní se tahalo po různých minulých zpěvnících a vlastních
materiálech, a tak texty písní nebyly každému dostupné.
Proto v roce 2017 generace kluků z Pražské Dvojky, v té době pod vedením Jindry
a~Docenta, dospěla k tomu, že zpěvníková situace není uspokojivá, a jali se
vytvořit trvalejší zpěvník, jenž by obsahoval všechny oblíbené písničky.
Albert tehdy založil
repozitář pro tento zpěvník a~po tvrdé práci spolu s~ostatními členy v~jazyku
\LaTeX \, při předtáborové LAN party, vytvořili prototyp zpěvníku následně
přetvořeného zejména Jindrou do zpěvníku 2018. Díky {\LaTeX}u byla tato verze
automaticky formátována a přizpůsobena pro A5 kroužkovou vazbu umožňujíce
vkládat nové písně do stejného zpěvníku. Paralelně s kluky Sluníčka vytvořila
vlastní zpěvník, a tak problém nedostupnosti písní pro každého na našich
společných táborech přetrvával, dokud se zpěvníky nespojily do společného r.
2020, přibližně dvojnásobně většího než zpěvník 2012. (Jindra taktéž vytvořil
paralelní verzi zpěvníku pro Wakany.)

Proč zpěvník 2020 teď nestačí? Protože minulý zpěvník byl tak trochu ušit
horkou jehlou. Při hraní písní jsme ve zpěvníku nalezli mnoho chyb a stále
chybělo několik oblíbených písní. V této verzi zpěvníku 2022 jsme přidali 18
nových a odstranili 23 nehraných písní oproti verzi 2020.

Albert si všiml, že mnoho kytaristů hraje jen jednoduché kvintakordy, což
většinou stačí, ale neví si rady s basovými a s rozšířenými akordy.
Zpěvník 2022 toto řeší přidáním hmatníkových diagramů akordů.
Nicméně, u některých písní diagramy akordů přímo neodpovídají (naschvál) jednoduchým
kvintakordům v písni. Tyto akordy slouží jako návrh k vylepšení nebo k
přiblížení hry k originálu.
Navíc, písně občas mají přidané taby s notami obsahující všelijaké vyhrávky,
a tak se do hraní může přidat i jiný nástroj než jen kytara.

\noindent
Velmi moc uvítáme jakékoli připomínky a nové návrhy.
Směřujte je prosím na GitHub\footnote{https://github.com/JindrazPrahy/DvojkarskyZpevnik}. %napsal jsem to ručně, aby to mohli lidi najít z fyzického výtisku, nedávám to do \url, aby to mělo stejný font jako zbytek textu
Zpěvník je vyvíjen open--source, a~tak každý může vzít zdrojový kód zpěvníku,
upravit ho nebo přidat něco nového a změny nahrát do repozitáře na GitHubu.\\

\noindent
Hodně zábavy a mnoho transcendentních zážitků při zpěvu přejí tvůrci zpěvníku\\
\textbf{Jindra} \& \textbf{Albert}.\\
26.\,06.\,2022

\vspace*{.75cm}

\noindent
Za úvodní kresbu na přebalu děkujeme \textbf{Pavoukovi}.\\\\

\noindent
{\tiny \rmfamily Zpěvník 22.06, Děravý dikobraz LTS}

\end{abstract}
\newpage

% Aby nebylo na predmuve nic vzadu.
\phantom{.}
\newpage
\checkandfixthelayout

%%%%%%%%%%%%% Písně %%%%%%%%%%%%%%%%
%% Okraje:
% P+L = 4cm
% T = 1.5cm
% B = 0 cm

\newgeometry{top=1.5cm, bottom = 0cm, left = 2cm, right = 2cm}
\setlrmarginsandblock{1cm}{3cm}{*} %Kvůli dírám pro kroužky
\setulmarginsandblock{1.5cm}{0cm}{*}
\checkandfixthelayout 

% PRIDAVANI

\newcommand{\importsong}[2]{\phantomsection\addcontentsline{toc}{section}{#1}\input{../songy/#2}\newpage}

\pagestyle{simple}
\importsong{Čtyři slunce}{CtyriSlunce.tex}
\importsong{Indiáni ve městě}{IndianiVeMeste.tex}
\importsong{Já s tebou žít nebudu}{JaSTebouZitNebudu.tex}
\importsong{John Brown's Body}{JohnBrownsBody.tex}
\importsong{Marie}{Marie.tex}
\importsong{Muchomůrky bílé}{MuchomurkyBile.tex}
\importsong{Na co nesmíš zapomenout}{NaCoNesmisZapomenout.tex}
\importsong{Nic není tak horký}{NicNeniTakHorky.tex}
\importsong{O malém rytíři}{OMalemRytiri.tex}
\importsong{Pod horou}{PodHorou.tex}
\importsong{Racci}{Racci.tex} % 2-stranna, ale nemusi byt nutne hratelne bez otoceni
\importsong{Sfinga a Málek}{SfingaAMalek.tex}
\importsong{Take On Me}{TakeOnMe.tex}
\importsong{Taši delé}{Tasidele.tex}
\importsong{Tři Kříže}{TriKrize.tex}
\importsong{Za teplých letních nocí}{ZaTeplychLetnichNoci.tex}
\importsong{Známka punku}{ZnamkaPunku.tex}
\importsong{Ženy mužů}{ZenyMuzu.tex}


\pagestyle{empty}

%Aby byla titulni strana a strana s akordy na coveru
%+ pro tisk musí být počet stran dělitelný 4. (Aby byl
%celočíselný počet papírů, kde každý papír má 4 A5.

\phantom{.}
\newpage

%%%%%%%%%%%%% Přehled akordů %%%%%%%%%%%%%%%%
\newgeometry{top=0cm, bottom = 0cm, left = 0cm, right = 0cm}
\thispagestyle{empty}
\begin{figure}[h]
\centering
% PŘEHLED
{\hspace*{-1.5cm}
\includegraphics[height=\textheight]{../Akordy/AAAkordy3.pdf}
}
\end{figure}


\end{document}
