\normalsize
\sffamily
% PŘEDMLUVA
\lettrine{D}{}ržíš v rukou další z řady oddílových zpěvníků \textbf{Pražské Dvojky} a \textbf{Sluníček}.
Obsah zpěvníku tvoří jak historické Dvojkařské písně z minulého století
tak i jen pár let existující aktuální hity, na něž jsme museli z odposlechu
skládat akordy a noty.

Příběh oddílového zpěvníku sahá až do písní minulých generací Dvojkařů: do
zpěvníku 2004 od Aleše a Sumce, do roverského zpěvníku 2012 od Krdlíka,
do~\uv{Old--School} zpěvníku 2015 od Sumce a do minulých verzí tohoto zpěvníku
2018 a 2020. Tyto zpěvníky jsou však dnes k~nalezení pouze v našich srdcích.

Do Dvojky a do Sluníček začaly přicházet nové generace kluků a holek.
Časem byly objevovnány nové i starší hudební klenoty.
Mnoho písní se tahalo po různých minulých zpěvnících a vlastních
materiálech, a tak texty písní nebyly každému dostupné.
Proto v roce 2017 generace kluků z Pražské Dvojky, v té době pod vedením Jindry
a~Docenta, dospěla k tomu, že zpěvníková situace není uspokojivá, a jali se
vytvořit trvalejší zpěvník, jenž by obsahoval všechny oblíbené písničky.
Albert tehdy založil
repozitář pro tento zpěvník a~po tvrdé práci spolu s~ostatními členy v~jazyku
\LaTeX \, při předtáborové LAN party, vytvořili prototyp zpěvníku následně
přetvořeného zejména Jindrou do zpěvníku 2018. Díky {\LaTeX}u byla tato verze
automaticky formátována a přizpůsobena pro A5 kroužkovou vazbu umožňujíce
vkládat nové písně do stejného zpěvníku. Paralelně s kluky Sluníčka vytvořila
vlastní zpěvník, a tak problém nedostupnosti písní pro každého na našich
společných táborech přetrvával, dokud se zpěvníky nespojily do společného r.
2020, přibližně dvojnásobně většího než zpěvník 2012. (Jindra taktéž vytvořil
paralelní verzi zpěvníku pro Wakany.)

Proč zpěvník 2020 teď nestačí? Protože minulý zpěvník byl tak trochu ušit
horkou jehlou. Při hraní písní jsme ve zpěvníku nalezli mnoho chyb a stále
chybělo několik oblíbených písní. V této verzi zpěvníku 2022 jsme přidali 18
nových a odstranili 23 nehraných písní oproti verzi 2020.

Albert si všiml, že mnoho kytaristů hraje jen jednoduché kvintakordy, což
většinou stačí, ale neví si rady s basovými a s rozšířenými akordy.
Zpěvník 2022 toto řeší přidáním hmatníkových diagramů akordů.
Nicméně, u některých písní diagramy akordů přímo neodpovídají (naschvál) jednoduchým
kvintakordům v písni. Tyto akordy slouží jako návrh k vylepšení nebo k
přiblížení hry k originálu.
Navíc, písně občas mají přidané taby s notami obsahující všelijaké vyhrávky,
a tak se do hraní může přidat i jiný nástroj než jen kytara.

\noindent
Velmi moc uvítáme jakékoli připomínky a nové návrhy.
Směřujte je prosím na GitHub\footnote{https://github.com/JindrazPrahy/DvojkarskyZpevnik}. %napsal jsem to ručně, aby to mohli lidi najít z fyzického výtisku, nedávám to do \url, aby to mělo stejný font jako zbytek textu
Zpěvník je vyvíjen open--source, a~tak každý může vzít zdrojový kód zpěvníku,
upravit ho nebo přidat něco nového a změny nahrát do repozitáře na GitHubu.\\

\noindent
Hodně zábavy a mnoho transcendentních zážitků při zpěvu přejí tvůrci zpěvníku\\
\textbf{Jindra} \& \textbf{Albert}.\\
26.\,06.\,2022

\vspace*{.75cm}

\noindent
Za úvodní kresbu na přebalu děkujeme \textbf{Pavoukovi}.\\\\

\noindent
{\tiny \rmfamily Zpěvník 22.06, Děravý dikobraz LTS}
