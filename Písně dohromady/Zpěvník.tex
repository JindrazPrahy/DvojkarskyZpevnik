%%%%%%%%%%%%%%%%%%%%%%%%%%%%%%%%%%%%%%%%%%%%%%%%%%%%%
%			HLAVIČKA								%
%%%%%%%%%%%%%%%%%%%%%%%%%%%%%%%%%%%%%%%%%%%%%%%%%%%%%
\documentclass[openany,12pt]{memoir}

\usepackage[utf8]{inputenc}   
\usepackage[czech]{babel}
\usepackage[T1]{fontenc}
\usepackage[top=1.5cm, bottom=2cm, left=2cm, right=2cm]{geometry}  % --> NASTAVENÍ OKRAJŮ
\usepackage{fancyhdr}
\usepackage{graphicx}
\usepackage{lmodern}
\usepackage{xwatermark}
\usepackage{xcolor}



%%%%%% Package na zpěvník
\usepackage[full]{leadsheets}%http://mirrors.nic.cz/tex-archive/macros/latex/contrib/leadsheets/leadsheets_en.pdf   --> dokumentace	
\definesongtitletemplate{empty}{} 
\setchords{
format = \bfseries,   %tučné akordy
minor = {mi},% 
input-notation = {german},%
output-notation = {german}%
}
\definesongtitletemplate{empty}{} 

\newlength{\drop}
\newwatermark[allpages,color=red!50,angle=0,scale=2, xpos=0,ypos=0]{\includegraphics[width=5cm]{obr/pozadi.jpg}} %--> dvojka na pozadí


%%%% Vlastní příkazy
\newcounter{Slokočet}   %Automatické číslování slok
\newcommand{\mezera}{\vspace*{0.5cm}}   %Horizontální odsazení slok
\newcommand{\stred}{5.2cm}   %%% Na zarovnání slok doprostřed
\newcommand{\refren}{\mezera \noindent \textbf{R:} } %refrén
\newcommand{\sloka}{\mezera \noindent \addtocounter{Slokočet}{1} \arabic{Slokočet}. } 	%sloka, která se automaticky čísluje
\newcommand{\ssloka}{\mezera \noindent}  % vlastní číslo sloky


%%%%%%%%%%%%%%%%%%%%%%%%%%%%%%%%%%%%%%%%%%%%%%%%%%%%%%
%			NÁVOD									 %
%%%%%%%%%%%%%%%%%%%%%%%%%%%%%%%%%%%%%%%%%%%%%%%%%%%%%%
%1. Věci v hlavičce IGNOROVAT
%2. Píseň psát do prostoru mezi \begin{verse} a \end{verse}
%3. další řádek se značí dvěma odsazeníma (= dvakrát stisknout enter)
%4. \refren vždy na začátku refrenu a \sloka na začátek sloky (automaticky se čísluje)
%  \ = alt gr + q ; [/] = alt gr f/g ; {/} = alt gr + b/n; ^ = alt gr + 3 + mezera
%Cokoliv napíšete do ^{  } se bude brát jako akord
%když se toto bude dotýkat nějakého slova (nebude mezi tím a slovem mezera)
%tak se akordy zjeví nad slovem
%Když se to nedotýká slova, tak akord lítá ve vzduchu a vytiskne se větší mezera
%První možnost je asi preferovanější
%5. Akordy stačí psát jen do první sloky, když se nezmění -- kytaristi to zvládnou

%<++>
\usepackage{subfiles}
%</++>

\begin{document}
\pagestyle{plain}

\begin{song}{title=\predtitle\centering Černej Pasažér \\\large Traband \vspace*{-0.5cm}}  %% sem se napíše jméno songu a autor

\moveright -.5cm \vbox{
\begin{centerjustified}
\nejvetsi

\begin{varwidth}[t]{0.48\textwidth}\setlength{\parindent}{\pindent}  %Varianta č. 2 --> Dva sloupce
\nejvetsi

\setcounter{Slokočet}{0}
\sloka
Mám ^{Dmi\z}kufr~plnej přebytečnejch

^{A\z }krámů

a mapu zabalenou do ^{Dmi}plátna.

Můj vlak však jede na opačnou

^{A\z }stranu

a moje jízdenka je dávno ^{Dmi\z }neplatná.

\mezera
\textbf{F  Dmi F  Dmi }

\sloka
Někde ve vzpomínkách stojí dům.

Ještě vidím, jak se kouří z komína.

V tom domě prostřený stůl,

tam já a moje rodina.

\sloka
Moje minulost se na mě šklebí

a srdce bolí, když si vzpomenu,

že stromy, který měly dorůst

k nebi,

teď leží vyvrácený z kořenů.

\mezera
\textbf{F  Dmi F  Dmi }

\refren
Jsem černej ^{B\z }pasažér,

^{C\z }nemám ^{F}cíl ani směr,

vezu se ^{B\z }načerno ^{C\z }životem a ^{F\z }nevím.

Jsem černej ^{B\z }pasažér,

^{C\z }nemám ^{F}cíl ani směr,

vezu se ^{B\z }odnikud ^{C\z }nikam a ^{A7\z }nevím,

kde skončím.

\end{varwidth}\mezisloupci \begin{varwidth}[t]{0.48\textwidth}\setlength{\parindent}{\pindent}  % V případě varianty č.2 jde odsud text do pravé části
\vspace*{0.41cm}
\sloka
Mám to všechno na barevný fotce

někdy z minulýho století.

Tu jedinou a pocit bezdomovce

si nesu s sebou jako prokletí.

\mezera
\textbf{F  Dmi F  Dmi }

\refren

\sloka
Mám kufr plnej přebytečnejch

krámů

a mapu zabalenou do plátna.

Můj vlak však jede na opačnou

stranu

a moje jízdenka je dávno neplatná.

\refren

\sloka
Ze Smíchovskýho nádraží

rychlík mě někam odváží.

Srdce jak těžký závaží

ze Smíchovskýho nádraží\elipsa\dots

\end{varwidth}   %Součást druhé varianty

\end{centerjustified}
}
\end{song}
\setcounter{Slokočet}{0}
\newpage
\begin{song}{title=\predtitle\centering Černobílý svět \\\large Totální nasazení \vspace*{-0.3cm}}
\begin{centerjustified}
\setcounter{Slokočet}{0}
\sloka
^{Dmi}Jsme loutky ^{F}bez nití

^{B}já i všech ^{C\z }patnáct druhů ^{Dmi}mých.

Bez smrti ^{F}bez žití

^{B\z }bloumáme ^{C\z }tupě po polích.

\refren
^{Dmi}Černá a ^{B\z }bílá je,

^{C\z }osm kroků tam a osm ^{Dmi}sem,

frontová ^{B\z }linie.

^{C}Na věky proti sobě ^{Dmi}jdem.

\sloka
^{Dmi}Stojí tu ^{F\z }opodál

^{B}náš ^{\z C}domnělý to ^{\z Dmi}nepřítel.~~

I když jsem ^{F\z }černý král,

^{B\z }nejsem svých ^{C\z}figur ^{\z Dmi}velitel.~~

\sloka
Vždy první na tahu

jsem ten kdo bitvu rozpoutá,

ač nemá odvahu.

Bílý a nešťastný jsem král

\refren

\sloka
Přestal jsem počítat,

kolikrát jsem padnul a zas vstal,

kolikrát dostal mat,

útočil nebo utíkal.

\sloka
Jsme jenom figury

ať král či pěšák beze jména

a hraní na vojáky

je vaše lidská doména.

\refren

\end{centerjustified}
\end{song}
\setcounter{Slokočet}{0}
\newpage
%\documentclass[../main.tex]{subfiles}

\begin{song}{title=\centering Do Ameriky \vspace*{-0.3cm}}  %% sem se napíše jméno songu a autor
\moveright \stred \vbox{      %Varianta č. 1  ---> Jeden sloupec zarovnaný na střed
\setcounter{Slokočet}{0}
\sloka
^{C}Do Ameriky jezděj Parníky,

když tam přijdeš, ^{Am}zdá se ti to ^{Dmi }všecko velik^{G}ý,

^{\phantom{S}}je to fakticky hodně praktický  

^{G7}přijít tam a umět ^{C}anglicky. 

\refren
^{C}Dobrou noc -- good night, ^{D7}výborně -- all right,

^{G}Conrad Weidt ^{G7}-- už je off ^{C}side 

^{C7}His Master's ^{F}Voice, Yankee Doo^{A7}dle,

máš ape^{D7}tit? (mám) dem si štrůdl -- ^{G7}do pusy

^{C}Sto kilo je cent -- ^{D7}patent je patent,

^{G}Husa v troubě ^{G7}--  happy ^{C}end.

\refren

\sloka
Mám na západě chajdu v Nevadě,

zlatou žílu v Arizoně, doly v Kanadě,

babička na mě šetří v Panamě

ačkoli tu žiju náramě!

\refren

\sloka
Co Buffalo Bill vrazil do kobyl,

za to by si bejval byl koupil automobil,

cowboy na koni už se nehoní,

přestože mu Fordka nevoní.


}
\end{song}
\setcounter{Slokočet}{0}

\newpage

\end{document}
