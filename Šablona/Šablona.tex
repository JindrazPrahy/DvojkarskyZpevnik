%%%%%%%%%%%%%%%%%%%%%%%%%%%%%%%%%%%%%%%%%%%%%%%%%%%%%
%			HLAVIČKA								%
%%%%%%%%%%%%%%%%%%%%%%%%%%%%%%%%%%%%%%%%%%%%%%%%%%%%%
\documentclass[openany,12pt]{memoir}

\usepackage[utf8]{inputenc}   
\usepackage[czech]{babel}
\usepackage[T1]{fontenc}
\usepackage[top=1.5cm, bottom=2cm, left=2cm, right=2cm]{geometry}  % --> NASTAVENÍ OKRAJŮ
\usepackage{fancyhdr}
\usepackage{graphicx}
\usepackage{lmodern}
\usepackage{xwatermark}
\usepackage{xcolor}



%%%%%% Package na zpěvník
\usepackage[full]{leadsheets}%http://mirrors.nic.cz/tex-archive/macros/latex/contrib/leadsheets/leadsheets_en.pdf   --> dokumentace	
\definesongtitletemplate{empty}{} 
\setchords{
format = \bfseries,   %tučné akordy
minor = {mi},% 
input-notation = {german},%
output-notation = {german}%
}
\definesongtitletemplate{empty}{} 

\newlength{\drop}
\newwatermark[allpages,color=red!50,angle=0,scale=2, xpos=0,ypos=0]{\includegraphics[width=5cm]{obr/pozadi.jpg}} %--> dvojka na pozadí


%%%% Vlastní příkazy
\newcounter{Slokočet}   %Automatické číslování slok
\newcommand{\mezera}{\vspace*{0.5cm}}   %Horizontální odsazení slok
\newcommand{\refren}{\mezera \noindent \textbf{R:} } %refrén
\newcommand{\sloka}{\mezera \noindent \addtocounter{Slokočet}{1} \arabic{Slokočet}. } 	%sloka, která se automaticky čísluje
\newcommand{\ssloka}{\mezera \noindent}  % vlastní číslo sloky


%%%%%%%%%%%%%%%%%%%%%%%%%%%%%%%%%%%%%%%%%%%%%%%%%%%%%%
%			NÁVOD									 %
%%%%%%%%%%%%%%%%%%%%%%%%%%%%%%%%%%%%%%%%%%%%%%%%%%%%%%
%1. Věci v hlavičce IGNOROVAT
%2. Píseň psát do prostoru mezi \begin{verse} a \end{verse}
%3. další řádek se značí dvěma odsazeníma (= dvakrát stisknout enter)
%4. \refren vždy na začátku refrenu a \sloka na začátek sloky (automaticky se čísluje)
%  \ = alt gr + q ; [/] = alt gr f/g ; {/} = alt gr + b/n; ^ = alt gr + 3 + mezera
%Cokoliv napíšete do ^{  } se bude brát jako akord
%když se toto bude dotýkat nějakého slova (nebude mezi tím a slovem mezera)
%tak se akordy zjeví nad slovem
%Když se to nedotýká slova, tak akord lítá ve vzduchu a vytiskne se větší mezera
%První možnost je asi preferovanější
%5. Akordy stačí psát jen do první sloky, když se nezmění -- kytaristi to zvládnou



\begin{document}
\pagestyle{empty} % nechceme číslování
\begin{song}{title=\centering Černobílý svět \\\normalsize Totální nasazení \vspace*{-0.3cm}}  %% sem se napíše jméno songu a autor
%\moveright 5.5cm \vbox{      %Varianta č. 1  ---> Jeden sloupec zarovnaný na střed
\begin{minipage}[t]{0.48\textwidth}\setlength{\parindent}{0.45cm}  %Varianta č. 2 --> Dva sloupce
%%%% Před nechtěnou variantu dej na začátek řádku % (a před chtěnou odstraň); u druhé varianty jsou celkem 3 řádky!

\sloka ^{Dmi}Jsme loutky ^{F} bez nití
 
^{B}já i všech ^{C} patnáct ^{F} ^{E}druhů ^{Dmi} mých

bez smrti bez žití

^{Dmi}bloumáme tupě po polích




\refren ^{Dmi}Černá a ^{B} bílá je

^{C}osm kroků tam a ^{Dmi} osm sem

frontová ^{B} linie

^{C}na věky proti ^{Dmi} sobě jdem

\sloka Stojí tu ^{F} opodál

^{B}náš domnělý ^{C} to ^{F} ^{E}nepří ^{Dmi} tel

i když jsem černý král

nejsem svých figur velitel



\sloka Vždy první na tahu

jsem ten kdo bitvu rozpoutá

ač nemá odvahu

býlý a nešťastný jsem král


\refren

\end{minipage}\begin{minipage}[t]{0.48\textwidth}\setlength{\parindent}{0.45cm}  % V případě varianty č.2 jde odsud text do pravé části

\sloka Přestal jsem počítat

kolikrát jsem padnul a zas vstal

kolikrát dostal mat

útočil nebo utíkal

\sloka Jsme jenom figury

ať král či pěšák beze jména

a hraní na vojáky je vaše lidská doména




\end{minipage}   %Součást druhé varianty
\setcounter{Slokočet}{0}
\end{song}

\end{document}